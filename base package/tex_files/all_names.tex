\documentclass{article}[letter, 12pt]
\usepackage[a4paper, total={16.59cm, 22.94cm}]{geometry}
\usepackage[spanish]{babel}
\usepackage{inputenc}[utf8]
\usepackage{xcolor}
\usepackage{hyperref}
\hypersetup{
    colorlinks=true,
    linkcolor=gray,
    filecolor=blue,      
    urlcolor=blue,
}
\newlength\tindent
\setlength{\tindent}{\parindent}
\setlength{\parindent}{0pt}
\renewcommand{\indent}{\hspace*{\tindent}}

\def\code#1{\texttt{#1}}
\def\pseudocode#1{\texttt{\color{gray}\small#1}}
\def\codename#1{\texttt{\color{gray}\small#1}}

\usepackage[default]{lato}
\usepackage[T1]{fontenc}
\usepackage{inconsolata}
\usepackage[T1]{fontenc}
\usepackage{fancyhdr}
\usepackage{makecell}


\pagestyle{fancy}
\renewcommand{\headrulewidth}{0pt}
\fancyhf{}

\fancypagestyle{plain}{
  \fancyhf{}  
}
\rhead{Documentación de R \textbf{\emph{\color{blue}{4.0.3}}}}
\lhead{\code{all.names} \code{\{base\}} \pseudocode{extraer.nombres}}
\rfoot{\thepage}

\usepackage[titles]{tocloft}
\setlength{\cftbeforesecskip}{-.2ex}

\makeatletter
\renewcommand{\maketitle}{\bgroup\setlength{\parindent}{0pt}
\begin{flushleft}
  \textbf{\@title}
  \@author
\end{flushleft}\egroup
}
\makeatother

\usepackage{titlesec}

\titleformat{\section}
{\color{gray}\normalfont\Large}
{\color{gray}\thesection}{1em}{}

\titleformat{\subsection}
{\color{gray}\normalfont}
{\color{gray}\thesubsection}{1em}{}


\begin{document}
	\title{\Huge{Funciones para encontrar todos los \textit{nombres} de una \textit{expresión}}}
	\maketitle
	\section{\color{gray}Descripción}
	\paragraph{}
Las funciones \code{all.names()} y \code{all.vars()} devuelven un vector de caracteres que contiene todos los \href{run:/Vocabulary.pdf}{\textit{nombres}} que aparecen en una \href{run:/Vocabulary.pdf}{\textit{expresión}} o en una \href{run:/Vocabulary.pdf}{\textit{llamada}} de \codename{R}.
\tableofcontents{}
\section{\color{gray}Forma de uso o sintaxis}
\indent\code{all.names(expr, functions = TRUE, max.names = -1L, unique = FALSE)}\\
\indent\code{all.vars(expr, functions = FALSE, max.names = -1L, unique = TRUE)}\\

\section{\color{gray}Argumentos}

\begin{tabular}{|p{1.75cm}|p{2.5cm}|p{10.75cm}}
	\multicolumn{3}{p{14cm}}{Tabla 1. Argumentos de las funciones de extracción de nombres de los componentes de las sentencias de \codename{R}} \\
	\hline
	\multicolumn{1}{p{1.75cm}}{Argumento} & \multicolumn{1}{p{2.5cm}}{\makecell[l]{Valor \\ esperado}} & \multicolumn{1}{p{10.75cm}}{Descripción} \\
	\hline 
	\multicolumn{1}{p{1.75cm}}{\pseudocode{expresión}} \\
	\multicolumn{1}{p{1.75cm}}{\code{expr}} & \multicolumn{1}{p{2.5cm}}{Una sentencia.} & \multicolumn{1}{p{10.75cm}}{\href{run:/Vocabulary.pdf}{\textit{Expresión}} o \href{run:/Vocabulary.pdf}{\textit{llamada}} desde la cual se extraerán los nombres.} \\
 	\multicolumn{1}{p{1.75cm}}{\pseudocode{funciones}} \\
	\multicolumn{1}{p{1.75cm}}{\code{functions}} & \multicolumn{1}{p{2.5cm}}{Un valor lógico.} & \multicolumn{1}{p{10.75cm}}{Valor lógico que indica si los nombres de la función deberán ser incluidos en el resultado.}\\ 
	\multicolumn{1}{p{1.75cm}}{\pseudocode{máximo.de.nombres}} \\
	\multicolumn{1}{p{1.75cm}}{\code{max.names}} & \multicolumn{1}{p{2.5cm}}{Un valor entero.} & \multicolumn{1}{p{10.75cm}}{El número máximo de nombres que serán devueltos. Un valor de \code{-1} indicará que el único límite en el número de nombres devueltos será el de la extensión máxima de los vectores en R.}\\
	\multicolumn{1}{p{1.75cm}}{\pseudocode{nombres.únicos}} \\
	\multicolumn{1}{p{1.75cm}}{\code{unique}} & \multicolumn{1}{p{2.5cm}}{Un valor lógico.} & \multicolumn{1}{p{10.75cm}}{Valor lógico que indica si los nombres duplicados deberán ser removidos del resultado.}\\ 
 	\hline 
\end{tabular}

\section{\color{gray}Detalles}
\paragraph{}
Las \href{run:/Vocabulary.pdf}{\textit{llamadas}}, las \href{run:/Vocabulary.pdf}{\textit{expresiones}} y los \href{run:/Vocabulary.pdf}{\textit{nombres}} son los tipos de objetos que constituyen las estructuras sintácticas básicas del lenguaje en \codename{R}; para más información, por favor consulte la página de ayuda sobre cada tipo de objeto, así como el tomo sobre \href{run:/Vocabulary.pdf}{\textit{la definición del lenguaje R}} de los manuales de la documentación.\par
\paragraph{}
Las funciones \code{all.names()} y \code{var.names()} son herramientas que permiten descomponer la sintaxis de un conjunto de sentencias válidas, aunque todavía no evaluadas, del lenguaje. La primera función identificará variables, funciones y operadores, mientras que la segunda solo identificará nombres de variables u objetos.
\section{\color{gray}Valor devuelto}
\paragraph{}
La función \code{all.names()} devolverá un vector de caracteres con los nombres de las variables, funciones y operadores que encuentre en una sentencia \href{run:/Vocabulary.pdf}{no evaluada} de código de \codename{R}.
\paragraph{}
La función \code{all.names()} devolverá un vector de caracteres con los nombres de las variables que encuentre en una sentencia \href{run:/Vocabulary.pdf}{no evaluada} de código de R.
\section{\color{gray}Referencias}
\paragraph{}
R Core Team. \href{https://cran.r-project.org/doc/manuals/R-lang.html}{\textit{R Language Definition}}. The R Manuals. Viena: The R Foundation, 2021.
\section{\color{gray}También véase}
\paragraph{}
\href{run:/Vocabulary.pdf}{\code{substitute}}\code{()} para reemplazar símbolos por valores en una expresión.
\section{\color{gray}Ejemplos}
\indent\code{all.names(expression(sin(x+y)))} \\
\indent\code{all.names(quote(sin(x+y)))  \# o una llamada} \\
\indent\code{all.vars(expression(sin(x+y)))} \\
\section{\color{gray}Código fuente}
\subsection{\color{gray} \code{all.names()}}
\indent\code{function(expr, functions = TRUE, max.names = -1L, unique =FALSE)} \\
\indent\indent\code{.Internal(all.names(expr, functions, max.names, unique))}\\
\subsection{\color{gray}\code{all.vars()}}
\indent\code{function(expr, functions = FALSE, max.names = -1L, unique = TRUE)} \\
\indent\indent\code{.Internal(all.names(expr, functions, max.names, unique))}\\

\section{\color{gray}Sobre la traducción}
\paragraph{}
La traducción al español de esta página de ayuda fue actualizada el 03 de mayo de 2021 y está basada en la documentación original de \codename{R} en inglés para la versión 4.0.3. La revisión técnica de esta página de ayuda todavía no ha sido realizada. Si deseas participar realizando la revisión técnica o sugiriendo mejoras gramaticales/ortográficas/estilísticas al texto, por favor dirígete a la página del proyecto en: \\\href{https://github.com/sicabi/documentacionR}{https://github.com/sicabi/documentacionR} para saber un poco más sobre este proyecto. Toda participación será atribuída a su autor.
\\
\par\noindent\rule{\textwidth}{0.4pt}
\centerline{Paquete \{\code{base}\} versión \textbf{\emph{4.0.3}} \href{run:/Vocabulary.pdf}{Índice}}
\end{document}