\documentclass{article}[letter, 12pt]
\usepackage[a4paper, total={16.59cm, 22.94cm}]{geometry}
\usepackage[spanish]{babel}
\usepackage{inputenc}[utf8]
\usepackage{xcolor}
\usepackage{hyperref}
\hypersetup{
    colorlinks=true,
    linkcolor=gray,
    filecolor=blue,      
    urlcolor=blue,
}
\newlength\tindent
\setlength{\tindent}{\parindent}
\setlength{\parindent}{0pt}
\renewcommand{\indent}{\hspace*{\tindent}}  
\def\code#1{\texttt{#1}}
\def\pseudocode#1{\texttt{\color{gray}\small#1}}

\usepackage[default]{lato}
\usepackage[T1]{fontenc}
\usepackage{inconsolata}
\usepackage[T1]{fontenc}
\usepackage{fancyhdr}
\usepackage{makecell}

\pagestyle{fancy}
\renewcommand{\headrulewidth}{0pt}
\fancyhf{}

\fancypagestyle{plain}{
  \fancyhf{}  
}
\rhead{Documentación de R \textbf{\emph{\color{blue}{4.0.3}}}}
\lhead{\code{all.names} \code{\{base\}} \pseudocode{extraer.nombres}}
\rfoot{\thepage}

\usepackage[titles]{tocloft}
\setlength{\cftbeforesecskip}{-.2ex}
\usepackage{lipsum}

\makeatletter
\renewcommand{\maketitle}{\bgroup\setlength{\parindent}{0pt}
\begin{flushleft}
  \textbf{\@title}
  \@author
\end{flushleft}\egroup
}
\makeatother

\usepackage{titlesec}

\titleformat{\section}
{\color{gray}\normalfont\Large\bfseries}
{\color{gray}\thesection}{1em}{}

\begin{document}
	\title{\Huge{Funciones para encontrar todos los \textit{nombres} de una \textit{expresión}}}
	\maketitle
	\section{\color{gray}Descripción}
	\paragraph{}
Las funciones \code{all.names()} y \code{all.vars()} devuelven un vector de caracteres que contiene todos los \href{run:/Vocabulary.pdf}{\textit{nombres}} que aparecen en una expresión o en una llamada.
\tableofcontents{}
\section{\color{gray}Forma de uso o sintaxis}
\code{all.names(expr, functions = TRUE, max.names = -1L, unique = FALSE)}\\
\code{all.vars(expr, functions = FALSE, max.names = -1L, unique = TRUE)}\\

\section{\color{gray}Argumentos}

\begin{tabular}{|p{1.75cm}|p{2.5cm}|p{10.75cm}}
	\multicolumn{3}{p{14cm}}{Tabla 1. Argumentos de las funciones de extracción de nombres de los componentes de las sentencias de R} \\
	\hline
	\multicolumn{1}{p{1.75cm}}{Argumento} & \multicolumn{1}{p{2.5cm}}{\makecell[l]{Valor \\ esperado}} & \multicolumn{1}{p{10.75cm}}{Descripción} \\
	\hline 
	\multicolumn{1}{p{1.75cm}}{\pseudocode{expresión}} \\
	\multicolumn{1}{p{1.75cm}}{\code{expr}} & \multicolumn{1}{p{2.5cm}}{Una sentencia.} & \multicolumn{1}{p{10.75cm}}{\href{run:/Vocabulary.pdf}{\textit{Expresión}} o \href{run:/Vocabulary.pdf}{\textit{llamada}} desde la cual se extraerán los nombres.} \\
 	\multicolumn{1}{p{1.75cm}}{\pseudocode{funciones}} \\
	\multicolumn{1}{p{1.75cm}}{\code{functions}} & \multicolumn{1}{p{2.5cm}}{Un valor lógico.} & \multicolumn{1}{p{10.75cm}}{Valor lógico que indica si los nombres de la función deberán ser incluidos en el resultado.}\\ 
	\multicolumn{1}{p{1.75cm}}{\pseudocode{máximo.de.nombres}} \\
	\multicolumn{1}{p{1.75cm}}{\code{max.names}} & \multicolumn{1}{p{2.5cm}}{Un valor entero.} & \multicolumn{1}{p{10.75cm}}{El número de nombres máximo que serán devueltos. Un \code{-1} indicará que el único límite en el número de nombres devueltos será el de la extensión máxima de los vectores en R.}\\
	\multicolumn{1}{p{1.75cm}}{\pseudocode{nombres.únicos}} \\
	\multicolumn{1}{p{1.75cm}}{\code{unique}} & \multicolumn{1}{p{2.5cm}}{Un valor lógico.} & \multicolumn{1}{p{10.75cm}}{Valor lógico que indica si los nombres duplicados deberán ser removidos del valor.}\\ 
 	\hline 
\end{tabular}

\section{\color{gray}Detalles}
\paragraph{}
Las \href{run:/Vocabulary.pdf}{\textit{llamadas}}, \href{run:/Vocabulary.pdf}{\textit{expresiones}} y \href{run:/Vocabulary.pdf}{\textit{nombres}} son los tipos de objetos que constituyen las estructuras sintácticas básicas del lenguaje R; para más información, por favor consulte la página de ayuda de cada tipo, así como el tomo \href{run:/Vocabulary.pdf}{\textit{La definición del lenguaje R}} de la documentación.
\paragraph{}
Las funciones \code{all.names()} y \code{var.names()} son utilidades sintácticas que permiten descomponer e identificar las partes de un conjunto de sentencias válidas del lenguaje. La primera función identifica variables, funciones y operadores, la segunda solo identifica nombres de variables u objetos.
\section{\color{gray}Valor devuelto}
\paragraph{}
La función \code{all.names()} devolverá un vector de caracteres con los nombres de las variables, funciones y operadores que encuentre en una sentencia sin evaluar de código de R.
\paragraph{}
La función \code{all.names()} devolverá un vector de caracteres con los nombres de las variables que encuentre en una sentencia sin evaluar de código de R.
\section{\color{gray}También véase}
\paragraph{}
\href{run:/Vocabulary.pdf}{\code{substitute}}\code{()} para reemplazar símbolos por valores en una expresión.
\section{\color{gray}Ejemplos}
\code{all.names(expression(sin(x+y)))} \\
\code{all.names(quote(sin(x+y)))  \# o una llamada} \\
\code{all.vars(expression(sin(x+y)))} \\
\par\noindent\rule{\textwidth}{0.4pt}
\centerline{Paquete \{\code{base}\} versión \textbf{\emph{4.0.3}} Índice}
\end{document}