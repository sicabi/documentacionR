\documentclass{article}[letter, 12pt]
\usepackage[a4paper, total={16.59cm, 22.94cm}]{geometry}
\usepackage[spanish]{babel}
\usepackage{inputenc}[utf8]
\usepackage{xcolor}
\usepackage{hyperref}
\hypersetup{
    colorlinks=true,
    linkcolor=gray,
    filecolor=blue,      
    urlcolor=blue,
}
\newlength\tindent
\setlength{\tindent}{\parindent}
\setlength{\parindent}{0pt}
\renewcommand{\indent}{\hspace*{\tindent}}

\def\code#1{\texttt{#1}}
\def\pseudocode#1{\texttt{\color{gray}\small#1}}
\def\codename#1{\textbf{\texttt{\color{gray}#1}}}

\usepackage[default]{lato}
\usepackage[T1]{fontenc}
\usepackage{inconsolata}
\usepackage[T1]{fontenc}
\usepackage{fancyhdr}
\usepackage{makecell}

\usepackage[titles]{tocloft}
\setlength{\cftbeforesecskip}{-.2ex}

\makeatletter
\renewcommand{\maketitle}{\bgroup\setlength{\parindent}{0pt}
\begin{flushleft}
  \textbf{\@title}
  \@author
\end{flushleft}\egroup
}
\makeatother

\usepackage{titlesec}

\titleformat{\section}
{\color{gray}\normalfont\Large}
{\color{gray}\thesection}{1em}{}

\titleformat{\subsection}
{\color{gray}\normalfont}
{\color{gray}\thesubsection}{1em}{}

\pagestyle{fancy}
\renewcommand{\headrulewidth}{0pt}
\fancyhf{}

\fancypagestyle{plain}{
  \fancyhf{}  
}
\rhead{Documentación de R \textbf{\emph{\color{blue}{4.0.3}}}}
\lhead{\code{Encoding} \code{\{base\}} \pseudocode{Codificación}}
\rfoot{\thepage}


\begin{document}
	\title{\Huge{Funciones para leer o declarar la codificación de un vector de caracteres}}
	\maketitle
\section{\color{gray}Descripción}
\paragraph{}\code{Encode()} es la función principal para leer, así como para declarar, la codificación de un vector de texto. En la forma \code{Encode(x)}, permite leer la codificación de caracteres de un vector de texto \textit{\code{x}} en caso de ésta haya sido previamente declarada. Alternativamente, en la forma \code{Encode(x) <- etiqueta}, permite declarar a una etiqueta reconocida por \codename{R} como la codificación del vector \code{x}. Véase más adelante la sección \textit{Detalles} para saber cuáles son las etiquetas aceptadas. Si desea conocer más sobre la definición general del concepto en computación de \textit{codificación de caracteres} véase la sección \textit{Historia y definición de la codificación de caracteres} en esta página de ayuda.\par
\paragraph{}Otras funciones, como \code{enc2native()}, declaran que la codificación de un vector de texto será la codificación predeterminada para la sesión actual de \codename{R}. La función \code{enc2utf8()} codifica a un vector de texto en el estándar UTF-8. Aunque son conceptos similares, es importante distinguir entre la codificación de los vectores de texto o tipo carácter, y la codificación de los archivos. Para conocer más sobre sus definiciones y sus diferencias véase la sección \textit{Codificación de archivos y codificación de vectores tipo caracter} en esta página de ayuda, así como la sección \textit{Codificación de conexión} en la página de ayuda referente a las conexiones en \codename{R}.\par
\tableofcontents{}
\section{\color{gray}Forma de uso o sintaxis}

\indent\code{Encoding(x)}\\

\indent\code{Encoding(x) <- etiqueta}\\

\indent\code{enc2native(x)}\\
\indent\code{enc2utf8(x)}\\

\section{\color{gray}Argumentos}

\begin{tabular}{|p{1.75cm}|p{2.5cm}|p{10.75cm}}
	\multicolumn{3}{p{14cm}}{Tabla 1. Argumentos de las funciones de extracción de nombres de los componentes de las sentencias de \codename{R}} \\
	\hline
	\multicolumn{1}{p{1.75cm}}{Argumento} & \multicolumn{1}{p{2.5cm}}{\makecell[l]{Valor \\ esperado}} & \multicolumn{1}{p{10.75cm}}{Descripción} \\
	\hline 
	\multicolumn{1}{p{1.75cm}}{\pseudocode{expresión}} \\
	\multicolumn{1}{p{1.75cm}}{\code{expr}} & \multicolumn{1}{p{2.5cm}}{Una sentencia.} & \multicolumn{1}{p{10.75cm}}{\href{run:/Vocabulary.pdf}{\textit{Expresión}} o \href{run:/Vocabulary.pdf}{\textit{llamada}} desde la cual se extraerán los nombres.} \\
 	\multicolumn{1}{p{1.75cm}}{\pseudocode{funciones}} \\
	\multicolumn{1}{p{1.75cm}}{\code{functions}} & \multicolumn{1}{p{2.5cm}}{Un valor lógico.} & \multicolumn{1}{p{10.75cm}}{Valor lógico que indica si los nombres de la función deberán ser incluidos en el resultado.}\\ 
	\multicolumn{1}{p{1.75cm}}{\pseudocode{núm.de.nombres}} \\
	\multicolumn{1}{p{1.75cm}}{\code{max.names}} & \multicolumn{1}{p{2.5cm}}{Un valor entero.} & \multicolumn{1}{p{10.75cm}}{El número máximo de nombres que serán devueltos. Un valor de \code{-1} indicará que el único límite en el número de nombres devueltos será el de la extensión máxima de los vectores en R.}\\
	\multicolumn{1}{p{1.75cm}}{\pseudocode{nombres.únicos}} \\
	\multicolumn{1}{p{1.75cm}}{\code{unique}} & \multicolumn{1}{p{2.5cm}}{Un valor lógico.} & \multicolumn{1}{p{10.75cm}}{Valor lógico que indica si los nombres duplicados deberán ser removidos del resultado.}\\ 
 	\hline 
\end{tabular}

\section{\color{gray}Detalles}

\paragraph{}Los vectores de tipo cadena de carácter (es decir, de texto) pueden codificarse con las etiquetas \textquotedbl\code{UTF-8}\textquotedbl, \textquotedbl\code{latin1}\textquotedbl, \textquotedbl\code{unknown}\textquotedbl\ o \textquotedbl\code{bytes}\textquotedbl, que corresponden, a su vez, a los estándares de codificación de texto UTF-8, Latin-1, ASCII y binaria (es decir, sin codificar). En términos generales, las codificaciones UTF-8 y Latin-1 son los dos estándares básicos admitidos para vectores de tipo carácter dentro de \codename{R} y corresponden a los utilizados en plataformas similares a Unix y a Windows, respectivamente. Además de las codificaciones básicas, \codename{R} utiliza la etiqueta \textquotedbl\code{unknown}\textquotedbl\ (\color{gray}{\textquotedbl\pseudocode{desconocida}\textquotedbl}\color{black}{) para identificar cadenas de texto compuestas solamente por caracteres ASCII.}\par

\paragraph{}Debido a que los códigos de caracteres ASCII son los mismos para UTF-8 y Latin-1, se asume que un vector con estos caracteres podría pertenecer originalmente a cualquiera de ellas y por ello se le asigna la etiqueta correspondiente al caso de codificación desconocida. La etiqueta \textquotedbl\code{bytes}\textquotedbl\ (recomendada sólo para uso experto) se refiere a cadenas de texto que representan datos sin codificar o binarios. Si desea consultar la codificación bajo la cual está configurada su sesión de \codename{R} use la función sobre la información de localización: \code{l10n\_info()}, que devuelve una lista con valores lógicos etiquetados con la codificación de caracteres vigente para su sesión.\par 

\section{\color{gray}Valor devuelto}

\section{\color{gray}Referencias}

\section{\color{gray}También véase}

\section{\color{gray}Ejemplos}


\section{\color{gray}Sobre la traducción}
\paragraph{}
La traducción al español de esta página de ayuda fue actualizada el 03 de mayo de 2021 y está basada en la documentación original de \codename{R} en inglés para la versión 4.0.3. La revisión técnica de esta página de ayuda todavía no ha sido realizada. Si deseas participar realizando la revisión técnica o sugiriendo mejoras gramaticales/ortográficas/estilísticas al texto, por favor dirígete a la página del proyecto en: \\\href{https://github.com/sicabi/documentacionR}{https://github.com/sicabi/documentacionR}, para saber un poco más sobre cómo contribuir. Toda contribución será atribuida a la persona que la realice.
\\
\par\noindent\rule{\textwidth}{0.4pt}
\centerline{Paquete \{\code{base}\} versión \textbf{\emph{4.0.3}} \href{run:/Vocabulary.pdf}{Índice}}
\end{document}